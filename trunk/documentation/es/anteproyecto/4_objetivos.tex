\section{Alcances del proyecto}
{
	El proyecto aqu� planteado, deber� abordar los temas referentes a la etapa de producci�n de animaciones, mas precisamente los aspectos concernientes con el trazado y dibujo.
	Para enfrentarnos a los problemas generados en esta etapa mediante la propuesta que estamos presentando, es necesario implementar un servidor de transacciones que gestione y sincronice la tarea de los diferentes actores involucrados en el proceso. Los datos de entrada de este servidor ser�n enviados por cada uno de los animadores, que trabajan en nodos distintos.
	\ds
	Asimismo, es necesario implementar una aplicaci�n cliente, que ser� la interfaz sobre la cual el animador trabajar�, esta aplicaci�n deber� contar con las suficientes caracter�sticas para que el animador pueda plasmar su arte.
	\ds
	El servidor deber� proveer un mecanismo de persistencia donde se almacene en un solo lugar toda la informaci�n, objetos, cuadros, escenas, etc�tera y los clientes podr�n recuperar esa informaci�n solicit�ndola al servidor de manera transparente.
	\ds
	Para que el sistema funcione bien, es necesario proveer un sistema seguro de concurrencia que permita bloquear las �reas u objetos de trabajo para que otras animadores no destruyan o modifiquen la labor que se esta realizando. De esta misma manera, es necesario restringir el acceso a los documentos para evitar que cualquier persona pueda modificar, da�ar o hurtar los mismos.
	\ds
	Como el sistema deber� permitir el trabajo desde diferentes lugares, es decir, no es necesario que los animadores se concentren en un mismo lugar a trabajar, el sistema deber� proveer un mecanismo de comunicaci�n para que los animadores puedan dialogar sobre los aspectos referentes al proyecto en curso, este mecanismo se implementara por medio de texto y no de voz.
	\ds
	Para concluir, el sistema deber� permitir el trabajo de diversos actores de la producci�n de animaciones simultaneamente en diversos sitios de trabajo, evitando la mayor�a de problemas de concurrencia que suelen ocurrir en estos casos.
}

\newpage
\section {Objetivos}
{
	\subsection{Objetivo general}
	{
		\begin{itemize}
		\item Desarrollar una herramienta para producir animaciones de forma colaborativa, a trav�s de la extensi�n de \hyperlink{ktoon}{KToon} y uso de la biblioteca \hyperlink{qt}{Qt},  para contribuir a la mejora y ahorro en los tiempos de producci�n de animaciones.
		\end{itemize}
	}
	
	\newpage
	\subsection{Objetivos espec�ficos}
	{
		\begin{itemize}
		\item Desarrollar un servidor de transacciones, apoyado en un protocolo de comunicaciones, para sincronizar clientes del servidor y gestionar la persistencia del proyecto.
		\item Definir el protocolo de comunicaci�n con el servidor, mediante el an�lisis de la arquitectura de la herramienta \hyperlink{ktoon}{KToon}, para establecer un est�ndar a seguir.
		\item Desarrollar un editor gen�rico extendido de componentes gr�ficos que permita al animador visualizar la creaci�n colaborativa de la animaci�n.
		\item Desarrollar un mecanismo de comunicaci�n o charla entre animadores, basado en el protocolo de comunicaciones, para permitir a los miembros del equipo la f�cil comunicaci�n.
		\item Analizar las caracter�sticas de los componentes gr�ficos de la aplicaci�n, sobre la actual arquitectura para abstraer los elementos mas importantes que ser�n enviados a todos los clientes del servidor.
		\item Desarrollar un sistema de persistencia remota administrada por el servidor, basada en la jerarqu�a del proyecto para evitar la redundancia de informaci�n entre los clientes del servidor.
		\end{itemize}
	}
	
	\newpage
	\subsection{Objetivos estrat�gicos}
	{
		\begin{itemize}
		\item Permitir la producci�n de animaciones mas r�pidamente, mediante el uso de la herramienta que se producir� al terminar este proyecto, contribuyendo as� a la disminuci�n en costos de tiempo y dinero.
		\item Permitir el uso de la herramienta desde diversos sitios de trabajo, mediante el sistema distribuido que brindara este proyecto, para acortar distancias y ahorrar en costos de transporte.
		\item Identificar errores para ayudar en la toma de decisiones sobre futuros proyectos.
		\end{itemize}
	}
}

