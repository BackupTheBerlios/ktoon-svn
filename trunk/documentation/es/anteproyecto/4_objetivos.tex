\section {Objetivos}
{
	\subsection{Objetivo general}
	{
		\begin{itemize}
		\item Desarrollar una herramienta para producir animaciones de forma colaborativa, a trav�s de la extensi�n de Ktoon y uso de la librer�a \hyperlink{qt}{Qt},  para contribuir a la mejora y ahorra en los tiempos de producci�n de animaciones.
		\end{itemize}
	}
	
	\newpage
	\subsection{Objetivos espec�ficos}
	{
		\begin{itemize}
		\item Desarrollar una servidor de transacciones, apoyado en un protocolo de comunicaciones, para sincronizar clientes del servidor y gestionar la persistencia del proyecto.
		\item Definir el protocolo de comunicaci�n con el servidor, mediante el an�lisis de la arquitectura de la herramienta ktoon, para establecer un est�ndar a seguir.
		\item Estudiar el problema de la concurrencia en este tipo de producciones, estudiando y analizando condiciones actuales en otros sistemas, para evitar posibles fallos en el sistema.
		\item Desarrollar un editor gen�rico extendido de componentes gr�ficos que permita al animador visualizar la creaci�n colaborativa de la animaci�n.
		\item Desarrollar un mecanismo de comunicaci�n o charla entre animadores, basado en el protocolo de comunicaciones, para permitir a los miembros del equipo la f�cil comunicaci�n.
		\item Definir las caracter�sticas de los componentes gr�ficos de la aplicaci�n, realizando un estudio sobre la actual arquitectura para abstraer los elementos mas importantes que ser�n enviados a todos los clientes del servidor.
		\item Desarrollar un sistema de persistencia remota administrada por el servidor, basada en la jerarqu�a del proyecto para evitar la redundancia de informaci�n entre los clientes del servidor.
		\end{itemize}
	}
	
	\newpage
	\subsection{Objetivos estrat�gicos}
	{
		\begin{itemize}
		\item Permitir la producci�n de animaciones mas r�pidamente, mediante el uso de la herramienta que se producir� al terminar este proyecto, contribuyendo as� a la disminuci�n en costos de tiempo y dinero.
		\item Permitir el uso de la herramienta desde diversos sitios de trabajo, mediante el sistema distribuido que brindara este proyecto, para acortar distancias y/o ahorrar en costos de transporte.
		\item Identificar errores para ayudar en la toma de decisiones sobre futuros proyectos.
		\end{itemize}
	}
}

