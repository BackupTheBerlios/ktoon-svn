\section{Aspectos Metodol�gicos}
{
	\subsection{Tipo de Proyecto y Justificaci�n de su Magnitud }
	{
		Fundamentalmente el proyecto es de tipo practico, con algunos aspectos investigativos, al final del proceso se entregara una herramienta con las funcionalidades propuestas, debido a la magnitud de los sub-productos que ser�n entregados son necesarias dos personas para realizar este trabajo de grado.
	}
	
	\subsection{Resultados esperados}
	{
		\begin{itemize}
		\item Una aplicaci�n servidor que servira para gestionar y centralizar la informaci�n.
		\item Una extensi�n de la aplicaci�n \hyperlink{ktoon}{KToon} que use el servidor como medio de gesti�n.
		\item Un documento que contenga el est�ndar de comunicaci�n con el servidor.
		\item C�digo fuente y ejecutable de las aplicaciones y bibliotecas desarrolladas en el proyecto, artefactos de an�lisis y de dise�o generados, documentaci�n interna.
		\end{itemize}
	}
	
	\subsection{Metodolog�a de desarrollo}
	{
		Para el desarrollo del proyecto se usara un proceso �gil, para esto se ha elegido la metodolog�a \hyperlink{xp}{\textit{Programaci�n Extrema}}, en combinaci�n con algunos artefactos propios de \hyperlink{uml}{\textit{UML}}.
		\ds
		El enfoque elegido nos permitir� definir unas caracter�sticas b�sicas de versiones a corto plazo, con la idea de implementar y corregir un conjunto de peque�os avances.
	}
	
	\subsection{Plan de Actividades}
	{
		La \hyperlink{xp}{Programaci�n Extrema} se compone de 4 fases importantes, que son: plan, dise�o, codificaci�n y pruebas, pero estas fases no se realizan de forma completa y en orden, sino que cada una se compone de actividades que se van desarrollando de manera iterativa e incremental.
		
		\subsubsection{Fase de planeaci�n}
		{
			\begin{itemize}
			\item Escribir el documento de anteproyecto de grado.
			\item Cada iteraci�n empieza con un plan, donde se escriben las tareas que cada iteraci�n debe cumplir.
			\item Analisis de riesgos.
			\end{itemize}
		}
		\subsubsection{Fase de dise�o}
		{
			\begin{itemize}
			\item Escribir el est�ndar de codificaci�n.
			\item Escribir las pol�ticas de uso del servidor de control de versiones.
			\item Por cada iteraci�n, realizar \hyperlink{refactorizacion}{refactorizaci�n} cuando sea necesario.
			\item Dise�o de pruebas.
			\item Definici�n de arquitectura y modulos.
			\end{itemize}
		}
		
		\subsubsection{Fase de codificaci�n}
		{
			\begin{itemize}
			\item Implementaci�n con avances r�pidos y optimizaciones al final.
			\item Codificaci�n de pruebas.
			\item Documentaci�n de usuario.
			\item Documentaci�n interna.
			\end{itemize}
		}
		
		\subsubsection{Fase de pruebas}
		{
			\begin{itemize}
			\item Aplicaci�n de pruebas unitarias.
			\item Cuando se encuentra un error de codificaci�n o de logica, se desarrollan pruebas para evitar volver a caer en el mismo.
			\end{itemize}
		}
		
		
		La documentaci�n de la tesis ser� una actividad que se desarrollara a lo largo de todo el proyecto y paralelo a las demas actividades.
	}
}

