\documentclass[letterpaper,10pt,spanish,titlepage]{article}

%Configuracion inicial
\PassOptionsToPackage{activeacute}{babel}

\usepackage[]{fontenc}
\usepackage[spanish]{babel}
\usepackage[latin1]{inputenc}
% \usepackage{moreverb}
\usepackage{epsfig}
\usepackage{fancyhdr}
\usepackage{multicol}
% \usepackage{rotating}
\usepackage{graphics}
\usepackage{graphicx}

\usepackage{float}
\usepackage{textcomp}
\usepackage{alltt}
\usepackage{times}
\usepackage{makeidx}
\usepackage{tocbibind}
\usepackage{array}

\newcommand{\TODO}[0] {
\begin{center}
	\textbf{\underline{\color{red}{Esta secci�n necesita ser completada.}}}
\end{center}
}

\ifx\pdfoutput\undefined
\usepackage[ps2pdf,
            pagebackref=true,
            colorlinks=true,
            linkcolor=blue
           ]{hyperref}
\usepackage{pspicture}
\else
\usepackage[pdftex,
%             pagebackref=true,
            colorlinks=true,
            linkcolor=blue
           ]{hyperref}
\fi

\bibliographystyle{alpha}
% \setcounter{tocdepth}{2}
% \setcounter{secnumdepth}{4}

\pagestyle{fancy}
\fancyhf{}

% configuraci�n del encabezado
\fancyhead[L]{
\includegraphics[width=7cm]{images/header.png}
}

\fancyhead[RO,LE]{\thepage}

% Tama�os
% \setlength{\textwidth}{15cm}
% \setlength{\topmargin}{3cm}
% \setlength{\headheight}{1cm}
% \setlength{\headsep}{0.8cm}
% \setlength{\textheight}{23cm}

%opening
\title
{
  \textsc{Universidad del Valle}\\
  Escuela de ingenier�a de sistemas y ciencias de la computaci�n \\
  \textbf{Modulo para la creaci�n de producciones de animaci�n mediante un enfoque colaborativo.}
}

\author
{
	David Alejandro Cuadrado Cabrera \\
	Jorge Humberto Cuadrado Cabrera \\
	\\
	\textbf{Directora:} Paola Johana Rodriguez \\
	\textbf{Asesor:} Carlos Andres Cajigas
}

\date{\today}

\begin{document}

\maketitle

\begin{abstract}

A continuaci�n se presenta una propuesta de trabajo de grado que trata de solucionar las problem�ticas en la producci�n de animaciones utilizando un enfoque \hyperlink{modelo_colaborativo}{colaborativo} para el proceso de creaci�n de las mismas.

Mediante este enfoque los \hyperlink{director_animacion}{directores de la animaci�n} podr�n ser capaces de fiscalizar el trabajo que lleva a cabo cada animador en la realizaci�n de los diferentes planos de animaci�nn, as� como gestionar todo el proceso de producci�n.

A su vez, ser� permitido que dos o m�s animadores interact�en en un mismo \hyperlink{scene}{plano} o \hyperlink{frame}{marco} si es
deseado, contribuyendo as� con mejoras notables en los tiempos de producci�n.

\end{abstract}

\tableofcontents

\section{Introducci�n}
{
	Durante muchos a�os la industria de producci�n de animaciones se ha regido por el proceso de animaci�n cuadro-a-cuadro, el cual se trata de dibujar todos los \hyperlink{frame}{cuadros} que componen la animaci�n, recientemente se han incorporado al proceso algunas t�cnicas avanzadas de \hyperlink{tweening}{interpolaci�n de movimiento} que simplifican el proceso.
	
	Con este proyecto se pretende facilitar la producci�n de animaciones mediante el enfoque tradicional.
}


\section{Planteamiento y Formulaci�n del Problema}
{
	A grandes rasgos, para crear una producci�n animada lo primero que se necesita es una idea proveniente del creador original, despues de que la idea esta clara es necesario realizar el gui�n de la producci�n, el cual consigna los di�logos y acciones que deben seguir los personajes de la producci�n.   
	
	Una vez que se ha dise�ado el gui�n, un miembro de la productora se encarga de realizar el story board, por lo general esta persona es el director del proyecto o el dise�ador de los personajes.  A partir del story board se construyen los sonidos necesarios y se procede a sincronizar las im�genes con el sonido, hasta este punto el proceso es conocido como pre-producci�n.
	
	Para realizar el proceso de producci�n, cada animador debe encargarse de una parte de la animaci�n, una vez terminada esta parte debe entregar el trabajo al director para que �l ligue todas las partes que conforman la producci�n.  Durante todos estos sub-procesos, se realizan tareas de revisi�n y se reitera con el fin de mejorar y perfeccionar la producci�n.
	
	
	Dentro del proceso de producci�n uno de los aspectos m�s complejos es dirigir y coordinar todo el proceso de intercalaci�n o Inbetweener que consiste en dibujar a partir de los cuadros claves los cuadros restantes en un movimiento.  Debido al gran esfuerzo y coordinaci�n de dibujar un promedio de 24 cuadros por segundo de animaci�n, las productoras utilizan formatos impresos que indican con detalle la gu�a que debe tener cada cuadro dibujado.  Estos formatos son llamados la hoja exposici�n y la hoja de planeaci�n de dise�o. La primera es la gu�a de cada cuadro de la animaci�n que esta dividida por segundos, cada segundo esta sincronizado por un audio (banda de sonido o voz de personaje) que previamente es marcado en este formato; la segunda son los formatos que tienen las poses principales o cuadros claves de una animaci�n, en este formato se introduce el dibujo clave, una descripci�n de la acci�n, sugerencias de movimiento, numero de cuadros involucrados entre cuadros principales, detalles, tiempo, referencia de sonido y gu�a de colores.
	
	Dado el contexto anterior, es f�cil ver que los problemas generados a partir de la coordinaci�n de las producciones son:
	\begin{itemize}
	\item Perdidas de tiempo en los procesos de producci�n.
% 	\item 
	\end{itemize}
	
	Dichos problemas producen un gran costo en cuanto a tiempo y dinero para la empresa.  Por ello, para atender a las probl�maticas antes planteadas se propone el presente trabajo de grado, a trav�s del cual se propone una soluci�n que intenta simplificar el proceso de producci�n de animaci�n, contribuyendo a mejorar la facilidad de gestion del proyecto y teniendo en cuenta los siguientes aspectos:
	
	\begin{itemize}
	\item Centralizaci�n de la informacion: Mantener la informacion en un punto central, de facil acceso a los animadores.
	\item Descentralizaci�n del manejo de la informacion: Gestionar la informacion desde cada nodo de trabajo de un determinado animador o director.
	\end{itemize}
}



\section{Justificaci�n}
{
	Con este proyecto de grado se pretende aplicar el modelo de trabajo colaborativo a la producci�n de animaciones. Se considera que este modelo funciona bien para este tipo de trabajos, teniendo en cuenta que los gr�ficos se hace de manera vectorial y que de esta manera se pueden enviar por red los diversos componentes gr�ficos del proyecto y mediante peticiones al servidor se podr�a crear progresivamente la jerarqu�a del proyecto de animaci�n.
	
	Con el uso de la herramienta a desarrollar, las empresas dedicadas a la producci�n de animaciones tendr�an un gran ahorro en el tiempo de producci�n, lo que implica ahoro de dinero para la organizaci�n.  Adem�s el modulo para la Administraci�n que se encargara de la asignaci�n de tareas y llevara una historia del proceso de producci�n, permitir� a los directores detectar errores en el proceso y ayudara en la toma de decisiones administrativas para el proyecto en curso o futuros proyectos.
	
	Este proyecto resuelve un problema real y latente para la industria de produccion de animaciones,
	implica aspectos relacionados con el desarrollo de software y puede constituir como una herramienta novedosa en su campo de accion.
}






\section{Alcances del proyecto}
{
	El proyecto aqu� planteado, deber� abordar los temas referentes a la etapa de producci�n de animaciones, mas precisamente los aspectos concernientes con el trazado y dibujo. \\
	Para enfrentarnos a los problemas generados en esta etapa mediante la propuesta que estamos presentando, es necesario implementar un \textbf{servidor de transacciones} que gestione y sincronice la tarea de los diferentes actores involucrados en el proceso. Los datos de entrada de este servidor ser�n enviados por cada uno de los animadores, que trabajan en nodos distintos.
	\ds
	El servidor deber� proveer un mecanismo de persistencia donde se almacene en un solo lugar toda la informaci�n, objetos, cuadros, escenas, etc�tera y los clientes podr�n recuperar esta informaci�n solicit�ndola al servidor de manera transparente.
	\ds
	Para que el sistema funcione bien, es necesario proveer un sistema seguro de concurrencia que permita bloquear las �reas u objetos de trabajo para que otras animadores no destruyan o modifiquen la labor que se esta realizando. De esta misma manera, es necesario restringir el acceso a los documentos para evitar que cualquier persona pueda modificar, da�ar o hurtar los mismos.
	\ds
	Asimismo, es necesario implementar una aplicaci�n cliente, que ser� la interfaz sobre la cual el animador trabajar�, esta aplicaci�n deber� contar con \nuevo{algunas} caracter�sticas de dibujo \nuevo{y deber� permitir extender o crear nuevas herramientas mediante plugins. \\
	
	Las herramientas b�sicas que debe contener la aplicaci�n son:
	\begin{itemize}
	\item Herramienta de trazado de lineas
	\item Herramienta para seleccionar, escalar y rotar gr�ficos.
	\item Modificar nodos de un gr�fico elegido.
	\item Herramienta para dibujar figuras geom�tricas, como lineas rectas, cuadrados, y c�rculos.
	\item Herramienta para a�adir texto a la animaci�n.
	\item Herramienta de ampliaci�n (Zoom)
	\end{itemize}
	
	De la misma manera, la aplicaci�n cliente deber� soportar al menos las siguientes caracter�sticas de dibujo:
	\begin{itemize}
	\item Cambio de color, grosor y textura de la linea.
	\item Cambio de color y textura de rellenos.
	\item Activaci�n y desactivaci�n del papel cebolla sobre el �rea de dibujo.
	\item Visualizaci�n y pre-visualizaci�n de la animaci�n.
	\item Hacer y rehacer acciones realizadas.
	\end{itemize}
	
	Respecto al cliente, con este proyecto de grado no se pretende construir una soluci�n definitiva para la creaci�n de animaciones, en principio no necesariamente se incluir�n todas las herramientas para construir animaciones de gran calidad, pero si soportar� su extensi�n, es decir, el cliente alg�n d�a podr� alcanzar la madurez necesaria para realizar animaciones de gran calidad.
	\ds
	Una tercera y m�s sencilla aplicaci�n ser� el cliente administrador, encargado de labores como gestionar usuarios, configuraci�n del servidor y copias de seguridad.}
	\ds
	Como el sistema deber� permitir el trabajo desde diferentes lugares, es decir, no es necesario que los animadores se concentren en un mismo lugar a trabajar, el sistema deber� proveer un mecanismo de comunicaci�n para que los animadores puedan dialogar sobre los aspectos referentes al proyecto en curso, este mecanismo se implementara por medio de texto y no de voz.
	\ds
	\nuevo{En sintesis, este trabajo desarrollara los siguientes aspectos:}
	\begin{itemize}
	\item El dise�o y desarrollo de un servidor que soporte todos los diversos proyectos que desarrollan los clientes.
	\item Proveer� interfaces seguras para su almacenamiento y manipulaci�n, sin permitir que ocurran da�os en los diversos proyectos.
	\item Se incorporar� interfaces que permitan el acceso al mismo desde diferentes sitios geogr�ficos.
	\item Se incorporaran herramientas de trabajo a nivel b�sico, para que se pueda seguir desarrollando hasta alcanzar un nivel alto de aplicaci�n.
	\end{itemize}
}

\newpage
\section{Objetivos}
{
	\subsection{Objetivo general}
	{
		\begin{itemize}
		\item \nuevo{Desarrollar una herramienta para producir animaciones de forma colaborativa, para contribuir a la mejora y ahorro en los tiempos de producci�n de animaciones.}
		\end{itemize}
	}
	
	\newpage
	\subsection{Objetivos espec�ficos}
	{
		\begin{itemize}
		\item Desarrollar un servidor de transacciones, apoyado en un protocolo de comunicaciones, para sincronizar clientes del servidor y gestionar la persistencia del proyecto.
		\item Definir el protocolo de comunicaci�n con el servidor, \nuevo{mediante el an�lisis de las posibles acciones que se pueden realizar}, para establecer un est�ndar a seguir.
		\item Desarrollar un editor gen�rico extendido de componentes gr�ficos que permita al animador visualizar la creaci�n colaborativa de la animaci�n.
		\item Desarrollar un mecanismo de comunicaci�n o charla entre animadores, basado en el protocolo de comunicaciones, para permitir a los miembros del equipo la f�cil comunicaci�n.
		\item Analizar las caracter�sticas de los componentes gr�ficos de la aplicaci�n, \nuevo{abstrayendo} los elementos mas importantes que ser�n enviados a todos los clientes del servidor.
		\item Desarrollar un sistema de persistencia remota administrada por el servidor, basada en la jerarqu�a del proyecto para evitar la redundancia de informaci�n entre los clientes del servidor.
		\end{itemize}
	}
	
	\newpage
	\subsection{Objetivos estrat�gicos}
	{
		\begin{itemize}
		\item Permitir la producci�n de animaciones mas r�pidamente, mediante el uso de la herramienta que se producir� al terminar este proyecto, contribuyendo as� a la disminuci�n en costos de tiempo y dinero.
		\item Permitir el uso de la herramienta desde diversos sitios de trabajo, mediante el sistema distribuido que brindara este proyecto, para acortar distancias y ahorrar en costos de transporte.
		\item Identificar errores para ayudar en la toma de decisiones sobre futuros proyectos.
		\end{itemize}
	}
}


\section{Marco de referencia}
{
	\subsection{Marco te�rico}
	{
	}
	
	\subsection{Antecedentes}
	{
	}
	
	\subsection{Marco conceptual}
	{
		\begin{itemize}
		\item Anim�tica \\
		Conocida tambi�n como tira leica o leica reel, es la grabaci�n y edici�n de las im�genes del  Story Board  sincronizada con la banda sonora y el audio de los personajes, con esta ficha es posible controlar los tiempos de la animaci�n y  entender en movimiento lo visualizado en el papel para corregir errores de ritmo poder generar una idea de como ser� la pel�cula al final.
		
		\item Backgrounds \\
		Fondos o escenograf�as de la animaci�n.
		
		\item Ciclo 
		Secuencia de animaci�n repetitiva para mostrar una acci�n como la de caminar o correr.
		
		\item Biblia \\
		El Universo del proyecto de animaci�n que contiene, el concepto, sinopsis  de los cap�tulos y el dise�o del personaje con indicaciones de  c�mo dibujarlo y animarlo.
		
		\item Exposure Sheets\\
		Hoja de exposici�n o la carta de rodaje donde se consigna con detalle cada uno de los cuadros que intervienen en una acci�n.
		
		\item Extremo\\
		Dibujo final, al final de una pose.
		
		\item Frame\\
		Es la unidad de una secuencia de animaci�n, en la televisi�n 30 frames  componen un segundo de animaci�n.
		
		\item Giros\\
		Las vistas frontales, laterales y de espaldas de un personaje permitiendo visualizar una imagen tridimensional del personaje.
		
		\item Gui�n \\
		En ingles llamado \textit{script} que es el desarrollo escrito de la acci�n, di�logo y  los sonidos de una animaci�n
		
		\item Hoja de Modelo \\
		Gu�a que representa todas las posiciones de las vistas girando en 360�, los detalles, y las diferentes poses, expresiones y estados de �nimo.
		
		\item Intercalado \\
		Tambi�n se llaman Intermedios o Inbetwenner, que son la cantidad de  dibujos que unen las puntas o dibujos claves en una animaci�n
		
		\item Key\\
		Es un dibujo clave representado por una pose importante en la animaci�n.
		
		\item Layout \\
		Es  la  planeaci�n gr�fica de un plano de animaci�n, donde se visualiza la composici�n b�sica y los movimientos de  una c�mara.
		
		\item Animaci�n Limitada \\
		La Animaci�n limitada es la industrializaci�n de la animaci�n, muy com�n en televisi�n donde se consumen series r�pidamente
		
		\item Lip-sync \\
		Es la \textit{sincronizaci�n} del audio con las im�genes. Y la descomposici�n del di�logo (fonemas) y la  m�sica en cuadros para acoplar imagen y audio.
		
		\item Mesa de animaci�n \\
		Mesa de luz donde se trabajan los fotogramas de una animaci�n, utilizando hojas de papel delgado, aprovechando las transparencias que deja la luz para visualizar el movimiento de la animaci�n.
		
		\item Prueba de L�nea \\
		Filmaci�n de la animaci�n en su primera etapa de realizaci�n. Llamada tambi�n \textit{Prueba de l�piz en 2D}. 
		\item Story-board \\
		Dibujos en forma de historieta de todo el desarrollo de la pel�cula, acompa�ado de los textos que explican las acciones y los di�logos de las escenas.
		
		\item Timming \\
		Ritmo. Visualizaci�n de los tiempos en animaci�n.
		\item Twinning \\
		Dibujar dos elementos (manos, brazos, piernas...) repitiendo la misma acci�n al mismo tiempo
		
		\item Creador Original \\
		El creador Original es la persona que plantea el concepto original de la historia, la base de la creaci�n. Puede ser un director, productor, ilustrador, novelista, o guionista. 

		\item Productora \\
		Es la compa��a que acoge el proyecto y decide invertir tiempo, dinero y recursos en todas las fases de la producci�n.
		
		\item Director  \\
		El Director es el responsable de todo  el desarrollo visual, art�stico y audiovisual del Proyecto. Es el l�der de la animaci�n  y determina el tipo de producci�n  que se desea realizar, visualiza el story board que son los dibujos detallados de las secuencias de animaci�n y  maneja toda la informaci�n que tiene que ver con el di�logo, la m�sica, el ritmo y el trabajo de c�mara.
		
		\item Productor \\
		Es la persona encargada de coordinar y seguir todas las tareas de de las personas involucradas en la producci�n, maneja los cronogramas y el presupuesto detallado y  junto al director decide  cosas importantes que ata�en a la producci�n.
		
		\item Director de Animaci�n \\
		Es un rol que esta entre el director y la producci�n, es el responsable de liderar el estilo de animaci�n y de revisar y supervisar la producci�n de todo el show desde la historia inicial, hasta la postproducci�n final.
		�l revisa los dibujos hechos por el equipo de animaci�n y visualiza el m�todo para realizar estas escenas.
		
		\item Dise�ador de Personajes \\
		Esta persona hace parte del equipo creativo de la producci�n y se involucra desde la pre-producci�n desarrollando todo el dise�o de los personajes para el proyecto. Este integrante del equipo provee una Biblia del proyecto con toda la informaci�n gr�fica de los personajes, detalladas en poses, gestos y construcci�n del personaje.
		
		\item Animadores Clave \\
		El animador clave trabaja desde el story board  y crea las escenas claves de la animaci�n, las posiciones del personaje, la integraci�n con los fondos y determina  el ritmo y la cantidad de frames asociados en el movimiento.
		
		\item Intercalador \\
		El  intercalador usa los cuadros claves de la animaci�n como punto referencial  para desarrollar los cuadros restantes, estos cuadros permiten generar una animaci�n mas fluida y  profesional. El intercalador no es un trabajo creativo y requiere much�simas horas de trabajo.
		
		\item Clean Up \\
		Como el intercalador es un trabajo que requiere muchas horas y se   encarga de darle detalle a la l�nea y de colorear todos los cuadros de la animaci�n.
		
		\item Storyboarder \\
		Es el ilustrador encargado de plasmar en dibujos cada uno de los planos de la animaci�n, por lo regular es el mismo director o el creador de los personajes del proyecto.

		\end{itemize}
	}
		
}


\section{Aspectos Metodol�gicos}
{
	\subsection{Tipo de Proyecto y Justificaci�n de su Magnitud }
	{
		Fundamentalmente el proyecto es de tipo practico, con algunos aspectos investigativos, al final del proceso se entregara una herramienta funcional con las funcionalidades propuestas, debido a la magnitud de los sub-productos que ser�n entregados son necesarias dos personas para realizar este trabajo de grado.
	}
	
	\subsection{Resultados Esperados}
	{
		\begin{itemize}
		\item Una aplicaci�n servidor que servira para gestionar y centralizar la informaci�n.
		\item Una extensi�n de la aplicaci�n KToon que use el servidor como medio de gesti�n.
		\item Un documento que contenga el est�ndar de comunicaci�n con el servidor.
		\end{itemize}
	}
	
	\subsection{Plan de Actividades}
	{
	}
}


\section{Cronograma}
{
	Ver anexo 1.
}


\section{Aspectos administrativos}
{
	A continuaci�n se listan los recursos necesarios para llevar a cabo este proyecto:
	\begin{itemize}
		\item{Dos estudiantes de ingenier�a de sistema para desarrollar el proyecto}
		\item{Un director de tesis, encargado de guiar a el proyecto.}
		\item{Un asesor, encargado de asesorar en temas de animaci�n a los estudiantes.}
		\item{Al menos dos equipos de c�mputo conectados entre s�, con su respectiva conexi�n a Internet.}
		\item{Transporte empleado durante el ciclo de desarrollo del proyecto. }
		\item{Fotocopias, libros, cederrones, lapices, borradores y material de oficina. }
	\end{itemize}
}


\section{Aspectos financieros}
{
	En la presente secci�n, se muestra un presupuesto estimada para el proyecto.
	\subsection{Presupuesto}
	{
		\begin{center}
		\begin{tabular}[c]{|l|l|l|}
			\hline
			\textbf{ITEM} & \textbf{CONCEPTO} & \textbf{VALOR}  \\ 
			\hline
			1.  	& 	Personal 		& 	\\
			1.1.  	& 	Desarrolladores 	& 	6'400.000 \\ 
			1.2.  	& 	Director de tesis 	& 	3'000.000 \\ 
			1.3.  	& 	Asesores 		& 	2'000.000 \\ 
			1.4.  	& 	Personal experto 	& 	1'000.000 \\
			\hline
			2. 	& 	Equipo de oficina 	& 	\\ 
			2.1.  	& 	Equipos de computo 	& 	1'500.000 \\ 
			2.2.  	& 	Documentaci�n 		& 	300.000 \\ 
			2.3.  	& 	CD's y DVD's 		& 	50.000 \\ 
			\hline
			3.  	& 	Generales 		& 	\\
			3.1	&	Telecomunicaciones	&	540.000 \\
			3.2.  	& 	Vi�ticos 		& 	1'440.000 \\ 
			4.  	& 	Imprevistos 		& 	500.000 \\
			\hline
			\hline
				& 	TOTAL 			& 	16'730.000 \\
			\hline
		\end{tabular}
		\end{center}
		
		
		\textbf{NOTAS}
		\begin{itemize}
		\item El presupuesto fue estimado sobre un tiempo de 8 meses.
		\end{itemize}
	}
}



\bibliography{bibliografia}

\end{document}

