\section{Introducci�n}
{
	Durante muchos a�os la industria de producci�n de animaciones se ha regido por el proceso de animaci�n cuadro-a-cuadro, el cual se trata de dibujar todos los \hyperlink{frame}{cuadros} que componen la animaci�n, recientemente se han incorporado al proceso algunas t�cnicas avanzadas de \hyperlink{tweening}{interpolaci�n de movimiento} que simplifican el proceso.
	\\
	Con este proyecto se pretende facilitar la producci�n de animaciones mediante el enfoque tradicional, as� como incrementar los posibles beneficios econ�micos que pueden generar las ventajas de tiempo que se provee al proceso con este proyecto.
	\\
	El proyecto esta enfocado directamente a las empresas desarrolladoras de producciones animadas, quienes  podr�n encontrar en este proyecto una soluci�n muy econ�mica a muchos problemas.
	
	
	\TODO
	Est� muy corto y no presenta claramente lo que implica la propuesta.  Sugiero indicar la manera tradicional de hacer animaciones (tal y como inician) y luego mostrar las debilidades de este proceso para finalizar indicando c�mo �stas ser�n abordadas a trav�s de la propuesta y los beneficios (claros y concisos) que acarrea para las empresas el uso de la herramienta a desarrollar.  Tratar de armar 1 p�gina.
}

