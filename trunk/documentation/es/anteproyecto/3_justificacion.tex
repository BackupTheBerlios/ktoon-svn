\section{Justificaci�n}
{
	Con este proyecto de grado se pretende aplicar el modelo de trabajo \hyperlink{modelo_colaborativo}{colaborativo} a la producci�n de animaciones. Se considera que este modelo funciona bien para este tipo de trabajos, teniendo en cuenta que los gr�ficos se hace de manera vectorial y que de esta manera se pueden enviar por red los diversos componentes gr�ficos del proyecto y mediante peticiones al servidor se podr�a crear progresivamente la jerarqu�a del proyecto de animaci�n.
	
	Con el uso de la herramienta a desarrollar, las empresas dedicadas a la producci�n de animaciones tendr�an un gran ahorro en el tiempo de producci�n, lo que implica ahoro de dinero para la organizaci�n.  Adem�s el modulo para la Administraci�n que se encargara de la asignaci�n de tareas y llevara una historia del proceso de producci�n, permitir� a los directores detectar errores en el proceso y ayudara en la toma de decisiones administrativas para el proyecto en curso o futuros proyectos.
	
	Este proyecto resuelve un problema real y latente para la industria de produccion de animaciones,
	implica aspectos relacionados con el desarrollo de software y puede constituir como una herramienta novedosa en su campo de accion.
}




