\section{Justificaci�n}
{
	Este proyecto pretende resolver un problema real y latente para la industria de producci�n de animaciones,
	implica aspectos relacionados con el desarrollo de software y puede constituir como una herramienta novedosa en su campo de acci�n.
	\ds
	As� mismo, gracias al uso de la herramienta las empresas dedicadas a la producci�n de animaciones deber�an tener un ahorro en el tiempo de producci�n, ya que se emplea un modelo de trabajo  \hyperlink{modelo_colaborativo}{colaborativo} para los procesos de la empresa, es decir, todos los actores del proceso trabajaran mancomunadamente para lograr sus metas y esto deber�a implicar un ahorro de dinero para la organizaci�n.
	\ds
	Finalmente, a la luz de la academia se justifica llevar a cabo est� proyecto ya que implicara la aplicaci�n de conceptos relacionados con \hyperlink{software}{Ingenier�a del Software}, bases de datos, computaci�n colaborativa, computaci�n gr�fica y lenguajes de programaci�n permitiendo a los estudiantes desarrolladores, profundizar, investigar y mejorar sus conocimientos en est�s �reas, las cuales son clave dentro de la formaci�n del Ingeniero de Sistemas.
}




