\section{Planteamiento y Formulaci�n del Problema}
{
	A grandes rasgos, para crear una producci�n animada lo primero que se necesita es una idea proveniente del creador original, despues de que la idea esta clara es necesario realizar el gui�n de la producci�n, el cual consigna los di�logos y acciones que deben seguir los personajes de la producci�n.   
	
	Una vez que se ha dise�ado el gui�n, un miembro de la productora se encarga de realizar el story board, por lo general esta persona es el director del proyecto o el dise�ador de los personajes.  A partir del story board se construyen los sonidos necesarios y se procede a sincronizar las im�genes con el sonido, hasta este punto el proceso es conocido como pre-producci�n.
	
	Para realizar el proceso de producci�n, cada animador debe encargarse de una parte de la animaci�n, una vez terminada esta parte debe entregar el trabajo al director para que �l ligue todas las partes que conforman la producci�n.  Durante todos estos sub-procesos, se realizan tareas de revisi�n y se reitera con el fin de mejorar y perfeccionar la producci�n.
	
	
	Dentro del proceso de producci�n uno de los aspectos m�s complejos es dirigir y coordinar todo el proceso de intercalaci�n o Inbetweener que consiste en dibujar a partir de los cuadros claves los cuadros restantes en un movimiento.  Debido al gran esfuerzo y coordinaci�n de dibujar un promedio de 24 cuadros por segundo de animaci�n, las productoras utilizan formatos impresos que indican con detalle la gu�a que debe tener cada cuadro dibujado.  Estos formatos son llamados la hoja exposici�n y la hoja de planeaci�n de dise�o. La primera es la gu�a de cada cuadro de la animaci�n que esta dividida por segundos, cada segundo esta sincronizado por un audio (banda de sonido o voz de personaje) que previamente es marcado en este formato; la segunda son los formatos que tienen las poses principales o cuadros claves de una animaci�n, en este formato se introduce el dibujo clave, una descripci�n de la acci�n, sugerencias de movimiento, numero de cuadros involucrados entre cuadros principales, detalles, tiempo, referencia de sonido y gu�a de colores.
	
	Dado el contexto anterior, es f�cil ver que los problemas generados a partir de la coordinaci�n de las producciones son:
	\begin{itemize}
	\item Perdidas de tiempo en los procesos de producci�n.
% 	\item 
	\end{itemize}
	
	Dichos problemas producen un gran costo en cuanto a tiempo y dinero para la empresa.  Por ello, para atender a las probl�maticas antes planteadas se propone el presente trabajo de grado, a trav�s del cual se propone una soluci�n que intenta simplificar el proceso de producci�n de animaci�n, contribuyendo a mejorar la facilidad de gestion del proyecto y teniendo en cuenta los siguientes aspectos:
	
	\begin{itemize}
	\item Centralizaci�n de la informacion: Mantener la informacion en un punto central, de facil acceso a los animadores.
	\item Descentralizaci�n del manejo de la informacion: Gestionar la informacion desde cada nodo de trabajo de un determinado animador o director.
	\end{itemize}
}


